\section{User Process}

Since the kernel process run whith full privileges, we need a user process to run programs executing safely, at lower privilege.
The user process is a mechanism to protect the operating system from malicious code. A program can do following bad things:
\begin{itemize}
  \item it can crash the operating system
  \item it can crash other user/kernel threads
  \item access control, to protect data and information of disk and/or other processes/threads
  \item incorrect usage e.g. of hardware
  \item TODO
\end{itemize}
How can you avoid these sorts of mistake, by:
\begin{itemize}
  \item memory protection
  \item limit the set of machine operations
  \item TODO
\end{itemize}
The user process will be loaded as a single, unseperated chunk in the physical memory.
To provide memory protection for user process, GeekOS uses segmentation.
A segment provides a private adress space inside the user process. This private adress space is used for the text and data segments of the process.



\section{Memory Management}

\subsection{Memory Segments}
A memory segment consists of a region of memory and the privilege level that is required to access the memory.





\subsection{Roadmap (this section is just copied)}
\begin{itemize}
  \item In src/geekos/user.c, you will implement the functions Spawn(), which starts a new user process
  \item and Switch\_To\_User\_Context(), 
		which is called by the scheduler before executing a thread in order to switch user address spaces if required.
  \item Exe\_Format data structure passed as a parameter. 
  		This data structure will be used by the Spawn() function to determine how to load the executable.
  \item In src/geekos/userseg.c, 
   		you will implement functions which provide support for the high level operations in src/geekos/user.c.
   		\subitem Destroy\_User\_Context(), frees the memory resources used by a user process
   		\subitem Load\_User\_Program() builds the User\_Context structure for a new process by loading parts of the executable program into memory.
   		\subitem The Copy\_From\_User() and Copy\_To\_User() functions copy data between the user address space and the kernel address space. 
   		\subitem The Switch\_To\_Address\_Space() function activates a user address space by loading the processor's LDT register with the LDT of a process.
   \item In src/geekos/kthread.c, you will implement functions which take a completed User\_Context data structure and create a thread which is ready to execute in user mode. 
   \item The Setup\_User\_Thread()
   \item Start\_User\_Thread() is a higher level operation which takes a User\_Context object and uses it to start
\end{itemize}
