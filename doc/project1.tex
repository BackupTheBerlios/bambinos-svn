\section{GeekOS Introduction}

\begin{itemize}
  \item GeekOS uses files to store executeables, these files are in ELF format.
  \item 
\end{itemize}

\subsection{dangling points}
In GeekOS loading ELF executeables is straight forward.
\begin{itemize}
  \item First you need to locate the ELF program headers
  \item you will fill in the fields of the \textit{Exe\_Format} data structure in the function \textit{Parse\_ELF\_Executable}
  \item 
\end{itemize}

\section{ELF}
Every ELF file consists of several \ldots 

%ELF now appears as the default binary format on operating systems such as Linux, Solaris 2.x,
%The three main types of ELF files are executable, relocatable, and shared object files

%Program headers are only important in executable and shared object files.
%The program header table is an array of entries where each entry is a 
%structure describing a segment in the
%object file or other information needed to create an executable process image.


%All sections in object files can be found using the Section header table.
%The section header, similar to the program header, is an array of structures




\subsection{Roadmap for project1}
\begin{itemize}
  \item check file integrity (if file is not NULL)
  \item verify correctness of the ``Magic Number'' (first 4 bytes of the ELF Header)%TODO what is the magic number and link 
  \item verify the maximum number of executable segments GeekOS allows  (defined in EXE\_MAX\_SEGMENTS)%TODO more
  \item fill in the struct: \textit{Exe\_Format}
\end{itemize}

