\documentclass[a4paper,10pt]{article}

\usepackage[T1]{fontenc}
\usepackage{color, listings}

\lstset{language=[Sharp]C, % Grundsprache ist C und Dialekt ist Sharp (C#)
basicstyle=\small,
captionpos=b, % Beschriftung ist unterhalb
%frame=lines, % Oberhalb und unterhalb des Listings ist eine Linie
frame=rlub
basicstyle=\ttfamily, % Schriftart
keywordstyle=\color{blue}, % Farbe für die Keywords wie public, void, object u.s.w.
commentstyle=\color{green}, % Farbe der Kommentare
stringstyle=\color{red}, % Farbe der Zeichenketten
breaklines=true, % Wordwrap a.k.a. Zeilenumbruch aktiviert
showstringspaces=false, % emph legt Farben für bestimmte Wörter manuell fest
emph={double,bool,int,unsigned,char,true,false,void},
emphstyle=\color{blue},
emph={Assert,Test},
emphstyle=\color{red},
emph={[2]\using,\#define,\#ifdef,\#endif}, 
emphstyle={[2]\color{blue}}
}

%\renewcommand{\rmdefault}{ppl} %Palatino
%\renewcommand{\rmdefault}{cmfib} % computer modern fibonacci 365
%\renewcommand{\rmdefault}{\sfdefault} default serifenlos
\usepackage{ae}
\renewcommand{\familydefault}{phv} % Helvetica angenehm zu lesen ! 386
\usepackage{graphicx}

% Silbentrennung
\usepackage[english,ngerman]{babel}

% PDF
%LaTeX erzeugt mit dem hyperref-Paket interaktive PDF-Dateien mit Bookmarks
\usepackage{color}
\usepackage[colorlinks]{hyperref}
\usepackage{my} % modifizierte Ueberschriften
%\usepackage{avant}
%\usepackage{mathptmx}

%empfohlen aber gehen leider nicht:
%\usepackage{microtype} Optischer Randausgleich besserer Rand
%\usepackage{mparhack}


%Einstellungen der Seitenränder
% \usepackage[left=3cm,right=3cm,top=3cm,bottom=3cm,includeheadfoot]{geometry}

%Umlaute ermöglichen
\usepackage[latin1]{inputenc}
\usepackage{tabularx} % Seite 259 im Latex Buch, ist extrem cool !!
\newcolumntype{Y}{>{\small\raggedright\arraybackslash}X}


%Kopf- und Fußzeile
\usepackage{fancyhdr}
\pagestyle{fancy}
\fancyhf{}

%Kopfzeile mittig
\fancyhead[C]{\nouppercase{\leftmark}}
%Linie oben
\renewcommand{\headrulewidth}{0.5pt}

%Fußzeile mittig
\fancyfoot[C]{\thepage}
%Linie unten
\renewcommand{\footrulewidth}{0.5pt}

\selectlanguage{english} 



\begin{document}

{\sc Die Schriftart small caps (Kapit"alchen).}
{\sf Die Schriftart sans serif (serifenlos).}

\tableofcontents
\newpage

TODO:

Welchen typen unterstuetzen wir
Alle Ausdruecke werden bei uns in der Klasse Symboltable ueberprueft auf Korrektheit. 


\section{Parser}

% TODO umschreiben (ANFANG)
We defined the first-sets and follow-sets to check that we can really achieve that a lookahead of one symbol is sufficient.

The first set very interesting for a top down analysis, because it determines the progress of the analysis.
The first set of a node contains all token, by which the parser will be guided to the node.

If the actual text isn't matched by one of the token of the first set, an other node will be chosen, namely the node with the first set that contains a token, which matches. If there is no such alternative the parsing is stopped with an error message. To guarantee that the decision between alternatives is definite, the first sets of the alternatives must be disjunctive, i.e.: there may not be a token contained in two alternative first sets. 

% TODO umschreiben (ENDE)

\subsection{First Sets}

% use packages: array
\begin{tabular}{p{4cm}l}
	S & First(S) \\ \hline
	& \\
	program & package\_declaration package\_import  \\
	package\_declaration & "package" \\
	package\_import & "import" \\
	class\_declaration & "public" \\
	class\_block & method\_declaration datatype\_declaration \\
	method\_declaration & "public" \\
	datatype\_declaration & "static" datatype\_decsriptor \\
	method\_call & identifier \\
	datatype\_assignment & identifier \\
	body\_block & while\_statement if\_statement return\_statement datatype\_declaration identifier (method\_call datatype\_assignment) \\
	while\_statement & "while" \\
	if\_statement & "if" \\
	return\_statement & "return" \\
	datatype\_decsriptor & datatype \\
	datatype & primitive object \\
	primitive & "int" "boolean" "char" \\
	object & String identifier \\
	identifier & letter \\
	selector & "." "[" \\
	condition & expression \\
	expression & term \\
	term & factor \\
	factor & value \\
	value & identifier datatype not\_value TODO \\
	integer & "-" digit\_non\_zero \\
	char & "'"\\
	boolean & "true" "false" \\
	string & """ \\
	digit & digit\_non\_zero digit\_zero \\
	digit\_non\_zero & 1 2 3 4 5 6 7 8 9 \\
	digit\_zero & 0 \\
\end{tabular}
\section{Memorymanagement in comPiler}
ComPiler compiles code for RISC-based architectures. It has 32 registers each
with size 32 Bit (4 Byte). Register 0 (R0) has always the value 0. Register 31
(R31) is reserved for return addresses. In a branch instruction the PC is stored
in R31. 

The instruction register (IR) holds the current instruction being executed. \newline 
The program counter (PC) contains the address of the instruction to be
fetched next. \newline
The stack pointer (SP) indicates the top element of a register-based stack. That
means that on some occasions registers are accessed like a stack and the SP
points out the next free register.  
\newline
The memory for the activation frames is organized like a stack. Each frame is an
entry. The SP indicates the next free memory and the FP the base address of the
current frame. 
\subsubsection{Organization of an activation frame}
An activation frame is a special memory context used for a procedure and its
local variables. It's created by a branch instruction (e.g. a procedure call). The base address of the activation 
frame is also the base address for all local variables declared here. Because this base address is highly 
important and one needs to access it efficently it's saved in the frame pointer (FP).
As the former PC was saved at the branch instruction, we use it as our return
address and save it in R31.



\subsection{local variables}
ComPiler offers local hiding of variables. That means, that procedures can
contain variables that cannot be seen outside the procedure. Even if there
exists a variable with the same name outside the procedure, these two don't
interfere. 
\subsubsection*{So what happens when a branch instruction occurs?}
When the instruction is executed, the PC is stored in R31. Then the code
generator jumps to the next instruction-address indicated by the branch instruction.
Now we have entered a special memory context for procedures (an activation frame). In this frame all memory is 
managed that is needed in the procedure
\newline
  
 


Local variables have the following properties: 
\begin{itemize}
  \item They have negative offsets to their baseaddresses
\end{itemize}



\section{Code Generation}

\subsection{Expressions}

Expressions act a central part in our compiler. Expressions are used for all kind of assignments, parameters, 
they are used for size declaration of arrays, for complex boolean and arithmetic expressions. 
Therefore expressions have to be very flexible and have to support all kinds of data types. 
The advantage of such a powerful method is that a programmer can be very flexible in writing code, 
the disadvantage is it is possible to create unstable code. Therefore the caller method has always to do a type check.
One of our focus, when we implemented our compiler was to be very reliable when compiling expressions.
We can compile and evaluate very long and complex expression, we have no limit at all in complexity :) .

At the end of an expression we return an Item which holds status information about the expression.
This information can be:
\begin{itemize}
  \item value of the expression is already in a Register (when a operation occurred)
  \item expression is a constant
  \item expression is a variable
  \item the return Item does always contain the type of the expression too, see \ref{labelTypeCheck}.
\end{itemize}
Boolean expressions are more complicated, see \ref{labelBoolean}. 

\subsubsection{Delayed codegeneration}
We support delayed code generation for arithmetic expressions. It is often unnecessary to load constants immediately to a register.
Consider following example $(3 + 4) * x$, without delayed code generation every constant is loaded into a register:
\begin{quote}
\begin{tabbing}
erste \= Spaltes\=  st sehr breit Spalte\=vierte \kill
\>ADDI \> 1,0,3\>R1:=3\\
\>ADDI \> 2,0,1\>R2:=4\\
\>ADD \>1,1,2\>R1:=R1+R2\\
\>LDW \> 1,0,x\>R2:=x\\
\>MUL \>1,1,2\>R1:=R1*R2\\
\end{tabbing}
\end{quote}
The same example with delayed code generation, $(3+4)$ will be calculated by the compiler (constant folding):
\begin{quote}
\begin{tabbing}
erste \= Spaltes\=  st sehr breit Spalte\=vierte \kill
\>LDW \> 1,0,x\>R1:=x\\
\>MULI \>1,0,7\>R1:=R1*7\\
\end{tabbing}
\end{quote}

\subsection{Boolean representation}
\label{labelBoolean}
We support a conditional and a repeated statement which both needs boolean expressions. Boolean expressions can be:
\begin{itemize}
  \item on of the key words \texttt{true} or \texttt{false},
  \item a boolean variable,
  \item a relation ($==$,$<=$,$>=$,$<$,$>$,$!=$),
  \item in addition boolean expressions can contain boolean operations AND ($ \&\& $)  and OR ($||$). 
\end{itemize}

\subsubsection{Default output of relations}
For simple expressions which contains a relation in it, we generator output code adequate for the AND operator:
The relation $(a<b)$ is leading to following output:
\begin{quote}
\begin{tabbing}
erste \= Spaltes\=  Jump\= Spaltes \kill
\>CMP\>a,b\\
\>BGE\>\>F(alse Jump)\\
\end{tabbing}
\end{quote}

\subsubsection{Nested boolean expression}
We implemented different levels for nested boolean expressions.
Consider following expression: 
\begin{quote}
$\underbrace{(\underbrace{\underbrace{(a<b)}_{3}\&\&\underbrace{(b<c)}_{3}}_{2})||(\ldots)}_1 $
\end{quote}
The operator $ \&\& $ has only permission to edit already generated output code of the same level or of deeper levels. When a
boolean operation has edit a certain output code $x$ the level of $x$ will be decreased. When $ \&\& $ is finished in the example above,
then all output code statements will be of the same level as $ \&\& $. This ensures that operations following the OR $||$ operator where the
$\ldots$ are can not edit previous output code.
That way we need brackets for a completed sub-expression and $||$ does not bind stronger than $ \&\& $. We did not improve this since we
have limited time for our compiler development.

\subsubsection{Editing rules for boolean operators}
\begin{enumerate}
  \item (default)
  \begin{quote}
	\begin{tabbing}
	xxxxxxxxxSpalteserste \= Spaltes\=  Jump\= Spaltes \kill
	$(a<b)$\>CMP\>a,b\\
	\>BGE\>\>F(alse Jump)\\
	\end{tabbing}
	\end{quote}
OR changes the precending control instruction when it has a false jump. It inverts the control instruction and changes the false into a true
  jump.
    \begin{quote}
	\begin{tabbing}
	xxxxxxxxxSpalteserste \= Spaltes\=  Jump\= Spaltes \kill
	$(a<b) || $\>CMP\>a,b\\
	\>BLT\>\>T(rue Jump)\\
	\end{tabbing}
	\end{quote}
\begin{verbatim}

\end{verbatim}
    \item
    
  \begin{quote}
	\begin{tabbing}
	xxxxxxxxxSpalteserste \= Spaltes\=  Jump\= Spaltes \kill
	$((a<b) \&\&  (b<d))$\>CMP\>a,b\\
	\>BGE\>\>F(alse Jump)\\
	\>CMP\>b,d\\
	\>BGE\>\>F(alse Jump)\\
	\end{tabbing}
	\end{quote}
      The operator OR changes the precending control instruction at position x, as described above. In addition it changes all False
      instructions at position < x to the programm counter next to $||$.  
  \begin{quote}
	\begin{tabbing}
	xxxxxxxxxSpalteserste \= Spaltes\=  Jump\= Spaltes \kill
	$((a<b) \&\&  (b<d))||$\>CMP\>a,b\\
	\>BGE\>\>next instruction after $||$\\
	\>CMP\>b,d\\
	\>BLT\>\>T(rue Jump)\\
	\end{tabbing}
	\end{quote}
\begin{verbatim}

\end{verbatim}

  \item   
  \begin{quote}
	\begin{tabbing}
	xxxxxxxxxSpalteserste \= Spaltes\=  Jump\= Spaltes \kill
	$((a<b) ||  (b<d))||$\>CMP\>a,b\\
	\>BLT\>\>T(rue Jump)\\
	\>CMP\>b,d\\
	\>BGE\>\>F(alse Jump)\\
	\end{tabbing}
	\end{quote}
The boolean operator OR does only change the precending control instruction as described in the first rule.
  \begin{quote}
	\begin{tabbing}
	xxxxxxxxxSpalteserste \= Spaltes\=  Jump\= Spaltes \kill
	$((a<b) ||  (b<d))||$\>CMP\>a,b\\
	\>BLT\>\>T(rue Jump)\\
	\>CMP\>b,d\\
	\>BLT\>\>T(rue) Jump)\\
	\end{tabbing}
	\end{quote}
\end{enumerate}
The AND operator works analogous, only true and false are inverted. The left false and true jumps which are not fixed at the end of the
expression will be fixed in the condition or repeated statements. Our rules works
really good, we can parse every expression independent of the complexity. We tested expressions with a level of 10 and more. \begin{small}
                                                          Idea copyright by team ComPiler :)
                                                          \end{small}
Worth notifying is that $\&\&$ works same as in Java when the first
expression is false the second one will not be evaluated. Analogous works $||$.
\subsection{Modules}
We support seperate compilation and we can call functions of a foreign class when the class is beeing imported.

\subsection{Methods}
Methods are one of the most important tool for structuring programms. The methods we can compile can have parameters and a return value. The
parameter can be of any kind of types (primitive,object). Our methods are able to hold local variables according to local hiding principle.
The steps a method goes through in our compiler:
\begin{itemize}
  \item When the parser recognize a method declaration the method gets an entry (cell $x$) in the global symbol table (\ref{labelSymboltable}).
  \item The return type will be recognized and stored as the method's type in the cell $x$. When a module serves as return
  type, then the module must have been imported. If no return type is declared a syntax errror will be printed.
  \item Next, the parameters are parsed and for each parameter an entry in the methods separate symbol table is created. The offset of the
  parameters is a positive value. 
  \item The methods prolog is executed  TODO LINK and the program counter of the prolog start is stored in cell $x$ of the global
  symboltable. The programmcounter is needed for further method calls. 
  \item Next, method's body will get parsed. All new symbols which are declared are stored in the sub- symbol list of the method. These
  local variables have a negative offset and is relative to cell $x$ of the method.
  \item At the end of the body we calculate the size of the method. We sum the size of the elements in the local symbol table and
  store the value into cell $x$ of the global symbol table.
  \item The methods epilog is executed. TODO LINK
  \item At this moment the method is ready to be called, recursive calls are possible too.
\end{itemize}
\begin{verbatim}

\end{verbatim}
\begin{quote}
Method's Prolog: \\
\begin{itemize}
  \item push return adress from the link register (R31) onto the stack
  \item push frame pointer (R29) onto the stack
  \item move frame pionter onto the current stack pointer
  \item decrease stack pointer with the size of the method (sum of local variables) 
\end{itemize}
\end{quote}
\begin{verbatim}

\end{verbatim}

\begin{quote}
Method's Epilog: \\
\begin{itemize}
  \item move the stack pointer (R30) to the current frame pointer. Thats the advantage of the stack, we just move the stack pointer and the
  memory is free.
  \item pop, write back the old frame pointer
  \item pop, write back the old return adress
  \item jump to the return adress and continue
\end{itemize}
\end{quote}
\begin{quote}
\begin{verbatim}

\end{verbatim}
Method Call: \\
\begin{itemize}
  \item put paramaters onto the stack
  \item branch to method
\end{itemize}

\end{quote}

\subsection{Arrays}
\label{labelArrays}
In our compiler an array is represented as several variables of the specified type, as many as the size of the array is. We implemented arrays
in an early stage of our compiler, at this stage we did not have moduls. The advantage of this implementation is that it is simple and
sufficient, the disadvantage is that we have no operations on arrays. \\
An array declaration has to be one this:
\begin{lstlisting}[caption={array declaration}]
dataType[] name = new dataType[expression];
dataType[] name = dataType[expression];
\end{lstlisting}
\paragraph{}When parsing an array declaration the type and size of the array will be stored in the symbol table. Arrays can be
local and global. To store new elements in the array they have to be, of course be of the same type. In addition we check we check the
index bounds when writing or accesing an element of an array at compile time.  






\section{Symbolfile}
The symbolfile (file-extension is .sym)is a sequential representation of the
symboltable. All entries
are representations of the symbols found in the sourcecode. \\
Because the symbolfile is just read one time during compilation, it cannot be
seen as a performance factor. So we decided to write the symbolfile in
xml-format as it is a good representation of objects. 
\subsection{Structure}
The root-tag of the symbolfile is \emph{<symbolfile>}. After the xml-header and
the root-tag the symbolfile starts with a list of the so-called \emph{module anchors}. Every
module anchor represents a module that's imported. In xml-syntax this lookes
like that:
\begin{lstlisting}[caption={module anchors}]
<modules>
	<module>
		<name> modulename 1 </name>
	</module>
	<module>
		<name> modulename 2 </name>
	</module>
	<module>
		<name> modulename 3 </name>
	</module>
</modules>
\end{lstlisting}

After the module anchors the actual symboltable representation starts. It is
enclosed in a \emph{<symbols>}-tag There are 3 different elementtypes that can be described here:
\begin{itemize}
  \item variables
  \item arrays
  \item methods 
\end{itemize}
\subsubsection{variables}
A variable has 2 properties: 
\begin{itemize}
  \item one of the primitive datatypes like \emph{boolean}, \emph{int} or \emph{char}
  \item the name of the variable in the sourcecode
\end{itemize}

\begin{lstlisting}[caption={variables}]
<variable>
	<type> type </type>
	<name> name </name>
</variable>
\end{lstlisting}

\subsubsection{arrays}
An array has 3 properties
\begin{itemize}
  \item one of the primitive datatypes like \emph{boolean}, \emph{int} or \emph{char}
  \item the name of the variable in the sourcecode
  \item the number of elements in the array
\end{itemize}

\begin{lstlisting}[caption={arrays}]
<array>
	<type> type </type>
	<name> name </name>
	<size> size </size>
</array>
\end{lstlisting}

\subsubsection{methods}
An array has 3 properties and contains another symboltable with the parameters
of the method. 
\begin{itemize}
  \item the return value of the method as one of the primitive datatypes
  like \emph{boolean}, \emph{int} or \emph{char}
  \item the name of the method in the sourcecode
  \item the number of elements in the array
\end{itemize}
So the description of a method is a recursive search through the symboltable.

\begin{lstlisting}[caption={methods}]
<method>
	<type> type </type>
	<name> name </name>
	<size> size </size>
	<symbols>
		<variable>
			<type> type </type>
			<name> name </name>
		<variable>
	</symbols>
</method>
\end{lstlisting}

\part{The linker}
The linker is a intermediate step between processing the sourcecode and executing a
binary. It combines objectfiles from the compiler to one binary for the virtual
machine. 
\\
A linker is needed if a compiler wants to provide \emph{separate compilation}.
That means that a program can import methods from a precompiled module. The
program just needs a symbolfile to find out which methods the module offers. \\
But a problem arises when the compiler wants to create a binary from the
program-sourcecode. The compiler knows that the called method exists in the
module (from the symbolfile) but it doesn't know the content of that method. \\
So before the program can be executed the linker has to copy the relevant data
from the affected objectfiles together into one binary. The objectfiles contain
all the information the linker needs to fullfill that work.    

\section{The objectfile}
\label{objectfile}
The objectfile is a intermediate file created by the compiler, that already
contains generated code and a lot of meta-information. The linker reads this
objectfile and creates a binaryfile. \\
The reason for that intermediate step is, that if modules are imported the
compiler doesn't have all the data he needs to create a valid and executable binary. 
So the objectfile offers the needed information like the program entry point, which
methods are loaded from modules and were to find that methods in these modules. \\
Based in that information the linker can link the compiled code in the
objectfiles together and create one executable binary.

\subsection{Structure}
\label{linker:objectfile:structure}
The structure of an objectfile is quite simple as one can see below. Its basic
datasize is 32 bit. That means all data except character are represented and 
stored as 32 bit integers. A character is a 1-byte-value (8-bit). \\
\begin{figure}[h]
	\begin{center}
	
		\begin{tabular}{|c|}
			\hline
			magic word \\
			\hline 
			branch instruction to main method \\
			\hline
			lenght of offset table \\
			\hline
			offset table \\ \\
			\hline
			lenght of fixup table \\
			\hline
			fixup table \\ \\
			\hline
			opCode \\ \\
			\hline 
		\end{tabular}
	
	\end{center}

\caption{structure of an objectfile}
\label{linker:objectfile:example:structure}

\end{figure} 

\subsubsection{magic word} 
\label{linker:objectfile:magic_word}
The magic word identifies the objectfile. Its value has to be $0$ represented
by a 32 bit integer value. 
\subsubsection{branch instruction to main method}
\label{linker:objectfile:branch_to_main}
This is an integer-representation of the branch-instruction to the address of
the main-method in this objectfile. 
\subsubsection{length of the offset table}
\label{linker:objectfile:length_offset_table}
The length of the offset table as a 32 bit integer value. The unit of this value
is one 32 bit word. 
\subsubsection{offset table}
\label{linker:objectfile:offset_table}
In the offset table variable- or methodnames and their offsets in the current
module are saved. That means that every exported module element stands in the
offset table with its name and opCode offset in the modulefile.  \\
The example in ~\ref{linker:objectfile:example:offset_table} shows the method
\emph{print}:

\begin{figure}[h]
	\begin{center}
		\begin{tabular}{|c|c|c|c|}
			\hline
			P &  R & I & N \\
			\hline
			T &  = &  0 & 0 \\
			\hline 
			\multicolumn{4}{|c|}{value} \\
			\hline
			\multicolumn{4}{|c|}{\ldots} \\
			\hline
		\end{tabular}
	\end{center}
	\caption{offset table}
	\label{linker:objectfile:example:offset_table}
\end{figure}

The bytes until the equal-sign form the name of the symbol. The next bytes
are skipped so that only complete 32 bit words are read (in our example 2 bytes). 
The next 32 bit word is the offset of the instruction (opCode) for access to that 
element in the module. The length of the offset table (~\ref{linker:objectfile:length_offset_table})  
defines how often this operations are performed. 

\subsubsection{length of the fixup table}
\label{linker:objectfile:length_fixup_table}
The length of the fixup table as a 32 bit integer value. The unit of this value
is one 32 bit word. 
\subsubsection{fixup table}
\label{linker:objectfile:fixup_table}
In the fixup table the imported symbols are listed. That means their modulename,
their name and their offset in the module are stored. \\
The example in ~\ref{linker:objectfile:example:fixup_table} shows a fixup table
with the \emph{print}-method from the module \emph{Util}.

\begin{figure}[h]
		
	\begin{center}
		\begin{tabular}{|c|c|c|c|}
			\hline
			U & T & I & L \\
			\hline
			. & P & R & I \\
			\hline
			N & T & = & 0 \\
			\hline 
			\multicolumn{4}{|c|}{value} \\
			\hline
			\multicolumn{4}{|c|}{\ldots} \\
			\hline
		\end{tabular}
	\end{center}
	
	\caption{fixup table}
	\label{linker:objectfile:example:fixup_table}
\end{figure}
The bytes from the beginning to the first dot form the modulename. The following
bytes to the equals-sign form the symbolname. The next bytes (in our example one
byte) are skipped so that only 32 bit words are read. \\
This reading-operation is performed until the number of words in the length of
the fixup table (~\ref{linker:objectfile:length_fixup_table}) is read.

\section{Processing of the objectfiles}
\label{linker:objectfile_procession}
As mentioned above the linker combines one or more objectfiles to one binary for
execution in the virtual machine.\\ 

\begin{figure}[h]

\begin{tabular}{c c}

		\begin{tabular}{|c|}
			\hline
			0 \\
			\hline 
			0 \\
			\hline
			0 \\
			\hline
			\\ \\ \\ \\ \\ \\
			\hline
			4 \\
			\hline
			% the fixup table
			\begin{tabular}{|c|c|c|c|}
				U & T & I & L \\
				\hline
				. & P & R & I \\
				\hline
				N & T & = & 0 \\
				\hline 
				\multicolumn{4}{|c|}{18} \\
			\end{tabular}
			% ! the fixup table
			\\
			\hline
			% the opCode
			\begin{tabular}{c}
				$ \cdots $ \\
				\hline
				line 18: method call for Util.print \\
				\hline
				$ \cdots $ \\
			\end{tabular}
			% ! the opCode
			\\
			\hline 
		\end{tabular}

		\begin{tabular}{|c|}
			\hline
			0 \\
			\hline 
			0 \\
			\hline
			6 \\
			\hline
			% offset table
			\begin{tabular}{|c|c|c|c|}
				P & R & I & N \\
				\hline
				T &  = &  0 & 0 \\
				\hline 
				\multicolumn{4}{|c|}{23} \\
				\hline
				C & A & L & C \\
				\hline
				= &  0 &  0 & 0 \\
				\hline 
				\multicolumn{4}{|c|}{42} \\
			\end{tabular}
			% ! offset table
			\\ 
			\hline
			0 \\
			\hline
			\\ \\ \\ \\
			\hline
			% the opCode
			\begin{tabular}{c}
				$ \cdots $ \\
				\hline
				line 23: entry point of method print \\
				\hline
				$ \cdots $ \\
			\end{tabular}
			% ! the opCode
			\\
			\hline 
		\end{tabular}

\end{tabular}
\caption{The objectfile of the main-module and the util-module}
\label{linker:example:linker}
\end{figure}

The example in ~\ref{linker:example:linker} shows what data the linker reads from the
objectfiles and how this data is processed. \\
The fixup table from the main-objectfile (left example) shows which information
must be read from the offset table of the imported objectfile (util-objectfile)
(right example). The offset-information from the fixup table determines the
position in the opCode that needs to be updated (18 in our example). 
That means: \\
Line 18 of the opCode in the main-module is a call of a method in another module
(module util, method print). 
\\
In the objectfile of that module (util) there is description where to find that
method. This description lies in the offset table. So the linker knows now where
the command that has to be fixed in the main-module lies and what information it
needs.  
\\
The information in the offset table is of course relative. How the linker
creates an absolute position for that method will be described later. 

\section{The binaryfile}
The binaryfile is an file that can be executed by the virtual machine. It hardly
provides any meta-information but just executable code (opCode). This file is
created when a linker links one or more objectfiles together. \\
The opCode is already created by the compiler and written into the objectfiles.
The linker copies that opCode together into a binaryfile and fixes
branchinformation in that opCode. \\
As we have heared before the opCode in the binary is addressed relatively. That
means that the first line of code in a module can be the 100th line of code in
the binary. So we need to fix that addressed, because otherwise
branch-instructions would point to incorrect addresses. 
 


\subsection{Structure}



\end{document}
