\section{Parser}

In our recursive-descent Parser we implemented all EBNF (see \label{labelEBNF}) production rules as
methods. The production rules of our EBNF allows alternatives, repetitions and optional parts.
%Figure (\ref{parser_deb}) shows a dependence diagram of the parser methods.  

To ensure that our EBNF is LL(1) conform, we created the a list with the first sets.
The parser compares the actual token with the first-sets symbol
and takes decisions between alternatives. 
To guarantee that the decision between two alternatives is correct, the first
set of the alternatives must be disjunctive. We checked our EBNF with the Parser tool: coco, to ensure we are are LL(1) conform.
% coco page http://www.ssw.uni-linz.ac.at/Research/Projects/Coco/

%\begin{figure}[h]
%\label{parser_deb}
%\begin{center}
%\includegraphics[width=13cm,height=13cm]{images/parser_edit.png}
%\end{center}
%\caption{parser methods dependencies}
%\end{figure}


\subsection{First Symbols(first sets)}
\label{first_sets}
\noindent
\begin{tabularx}{\linewidth}{YY}
program & ``\texttt{package}'' ``\texttt{import}'' ``\texttt{public}'' \\
packageDeclaration & ``\texttt{package}'' \\
packageImport & ``\texttt{import}'' \\
classDeclaration & ``\texttt{public}'' \\
identifier & simpleIdentifier \\
classBlock & simpleIdentifier ``\texttt{public}'' ``\texttt{String}'' ``\texttt{String[]}'' ``\texttt{int}'' ``\texttt{boolean}'' ``\texttt{char}'' ``\texttt{int[]}'' ``\texttt{boolean[]}'' ``\texttt{char[]}'' \\
objectDeclaration & simpleIdentifier \\
simpleDeclaration & ``\texttt{String}'' ``\texttt{String[]}'' ``\texttt{int}'' ``\texttt{boolean}'' ``\texttt{char}'' ``\texttt{int[]}'' ``\texttt{boolean[]}'' ``\texttt{char[]}'' \\
methodDeclaration & ``\texttt{public}'' \\
object & simpleIdentifier \\
objectDeclarationSuffix & simpleIdentifier \\
primitiveDeclaration & ``\texttt{int}'' ``\texttt{boolean}'' ``\texttt{char}'' \\
primitiveArrayDeclaration & ``\texttt{int[]}'' ``\texttt{boolean[]}'' ``\texttt{char[]}'' \\
stringDeclaration & ``\texttt{String}'' \\
stringArrayDeclaration & "String[]" \\
datatype & simpleIdentifier "String" "int" "boolean" "char" \\
datatypeDescriptor & simpleIdentifier "String" "int" "boolean" "char" \\
bodyBlock & simpleIdentifier "String" "String[]" "while" "if" "return" "int" "boolean" "char" "int[]" "boolean[]" "char[]" \\
primitive & "int" "boolean" "char" \\
assignmentSuffix & equal \\
primitiveArray & "int[]" "boolean[]" "char[]" \\
objectDeclarationAssignmentMethodcall & simpleIdentifier \\
arrayDeclarationSuffix & equal simpleIdentifier "[" \\
methodCallSuffix & "(" \\
expression & not number simpleIdentifier StringValue charValue "NULL" "-" "true" "false" \\
arraySelector & "[" \\
whileStatement & "while" \\
ifStatement & "if" \\
returnStatement & "return" \\
condition & not number simpleIdentifier StringValue charValue "NULL" "-" "true" "false" \\
value & not number simpleIdentifier StringValue charValue "NULL" "-" "true" "false" \\
intValue & number "-" \\
booleanValue & "true" "false" \\
factor & not number simpleIdentifier StringValue charValue "NULL" "-" "true" "false" \\
term & not number simpleIdentifier StringValue charValue "NULL" "-" "true" "false" \\
\end{tabularx}
%

\newpage
\subsection{Syntax Error Handling}
The parser is able to detect many syntax errors. We distinguish between two error levels : week- and strong error levels.
\begin{itemize}
  \item \textbf{Week Errors} don't do any harm. 
  \item \textbf{Strong Errors} will stop the code generation. 
\end{itemize}

\subsubsection{Week Errors (Warnings)}
\label{label_week_errors}
Week errors are missing tokens which will be inserted by the parser.
The code generator never takes notice about those errors. When a week error occurs a warning will be displaced, the token will be inserted, parsing and code generation continues.
Compiler is able to detect and correct following week errors:
\begin{itemize}
  \item any missing ";"
  \item any missing ")" "\{" "\}" "]" 
  \item near all missing "(" except in an object declaration, because we have to distinguish between "[" and "(" 
  \item in class declarations a missing ``class'' will be fixed.
  \item in method declarations a missing return type will be marked
  \item in method declarations when ``static'' is missing
  \item in parameterlists we will detect and correct missing ","
\end{itemize}

\subsubsection{Strong Errors}
\label{label_strong_errors}
Strong Errors can be syntax errors as well as invalid symbols or invalid words. When a strong error occurs the parser displaces a error
message and the code generation will be stopped. The parser goes to the next strong symbol and continue parsing. In our parser
implementation we have sync operation implemented in three methods. See Figure (\ref{parser_deb}), marked methods (TODO) serves as
synchronisation points. Synchronization works this way:
\begin{itemize}
  \item one of the synchronization methods recognizes a strong error in one of its called method
  \item the error will be printed and code generation will stop
  \item the synchronization methods x fetches new tokens until the next ``\hyperref[first_sets]{first set}'' token of method x is fetched
\end{itemize}
Strong Errors can be:
\begin{itemize}
  \item misplaced tokens. When the current token does not correspond to EBNF rules, except of week errors.
  \item any missing identifiers.
  \item any illegal terminal symbols, like identifier starting with a digit. This kind of error detects the scanner, and a
  error token will be delivered to the parser.
\end{itemize}
%\begin{quote}

Listing 2 shows a code example with several syntax errors in it, which will be marked by the compiler. The consolen output is shown below of listing.

\begin{figure}[h]
\label{synErrors}
%\begin{e}
\includegraphics[width=12.5cm,height=8cm]{images/synError.png}
%\end{center}
\end{figure}

%,label={synErrors}]
%\begin{lstlisting}[caption={Input language example}]
%package examples;
%
%import Util;
%
%public TestStuff {
%
%	int[] b = int[9]
%	
%	public static void compare(int a int b {
%		int x = 0;
%		int y = 1;
%		if (x==y {
%			println(x;
%		
%	}
%}
%\end{lstlisting}
The consolen output, when compiling the file in listing 2:
\begin{verbatim}
Warning: missing symbol: CLASS in line 5
Warning: missing symbol: NEW in line 7
Warning: missing symbol: SEMICOLON in line 9
Warning: missing symbol: COMMA in line 9
Warning: missing symbol: RPAREN in line 9
Warning: missing symbol: RPAREN in line 12
Warning: missing symbol: RPAREN in line 13
Warning: missing symbol: RBRACES in line 18
\end{verbatim}


\subsection {Type Savety}
\label{labelTypeCheck}

\subsubsection{Naming convention}
We ensure that variable names in a separate scope are unique. Local variables can have the same name as global variables. 

\subsubsection{Operator on incompatible Types}
\begin{itemize}
  \item will be ensured by EBNF
  \item 
\end{itemize}

\subsubsection{Typechecking}

\section{Symboltable}
\label{labelSymboltable}

The Symboltable is used to beware various information about symbols during compile time.
The Parser produces for every field (global variables, methods) and local variables an entry in the Symboltable.
We distinguish between different scopes. The global scope and local scopes. Every
method declaration has an entry in the gobal scope. The method's variables (local vars) are are stored in a sublist (local scope) of the corresponding method. TODO FIGURE

The Symboltable is representet by the symboltable list (SymbolTableList.java) which consists of cells (SymbolTableCell.java).
A cell holds following information:
\newline
\newline
\begin{tabular}{lp{6cm}}
\texttt{String name} & Name of the entry (Variable name, Method name,\ldots) \\
\texttt{ClassType classType} & kind of the entry: variable, array, procedure \\
\texttt{TypeDesc type} & data type: int, bool, char, String \\
\texttt{String value} & the value of the variable, when available \\
\texttt{SymbolTableList methodSymbols} & methods have a sub list with symbol table cells (local scope)\\
\texttt{int offset} & offset is negative \\
\texttt{int size} & in 4 bytes (word) \\
\texttt{int proc} & programm counter of methods (for fixups) \\
\texttt{boolean globalScope} & scope information \\
\end{tabular}

Imported Modules have their own Symboltable which are stored in seperate Container. For every imported Class file it's SymbolFile TODO LINK will be readed and a Symboltable created. To load a symbol from a imported Class file the name of the file must be specified, e.g. \texttt{Util.print()}.






