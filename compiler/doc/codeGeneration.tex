\section{Code Generation}

\subsection{Expressions}

Expressions act a central part in our compiler. Expressions are used for all kind of assignments, parameters, 
they are used for size declaration of arrays, for complex boolean and arithmetic expressions. 
Therefore expressions have to be very flexible and have to support all kinds of data types. 
The advantage of such a powerful method is that a programmer can be very flexible in writing code, 
the disadvantage is that it is possible to create unstable code. Therefore we have always to do a type checking after calling
the expression method.
One of our focus, when we implemented our compiler was to be very reliable in compiling expressions.
We can compile and evaluate very long and complex expression, we have no limit at all in complexity :) .

At the end of an expression we return an item which holds status information about the expression.
This information can be:
\begin{itemize}
  \item Type of the evaluated expression (constant, variable or register value)
  \item If the evaluated expression is a constant then the value is stored in the item too.
  \item The returned item contains the type of the evaluated expression too.
\end{itemize}
Compilation of boolean expressions is more complicated, see \ref{labelBoolean}. 

\subsubsection{Delayed code generation}
We support delayed code generation in arithmetic expressions. It is often unnecessary to load constants immediately onto registers.
Consider following example $(3 + 4) * x$, without delayed code generation every constant is loaded onto a register:
\begin{quote}
\begin{tabbing}
erste \= Spaltes\=  st sehr breit Spalte\=vierte \kill
\>ADDI \> 1,0,3\>R1:=3\\
\>ADDI \> 2,0,1\>R2:=4\\
\>ADD \>1,1,2\>R1:=R1+R2\\
\>LDW \> 1,0,x\>R2:=x\\
\>MUL \>1,1,2\>R1:=R1*R2\\
\end{tabbing}
\end{quote}
The same example with delayed code generation, $(3+4)$ will be calculated by the compiler (constant folding):
\begin{quote}
\begin{tabbing}
erste \= Spaltes\=  st sehr breit Spalte\=vierte \kill
\>LDW \> 1,0,x\>R1:=x\\
\>MULI \>1,0,7\>R1:=R1*7\\
\end{tabbing}
\end{quote}

\subsection{Boolean representation}
\label{labelBoolean}
We support a conditional and a repeated statement, both need boolean expressions. Boolean expressions can be:
\begin{itemize}
  \item one of the key words \texttt{true} or \texttt{false},
  \item a boolean variable,
  \item a result of a relation ($==$,$<=$,$>=$,$<$,$>$,$!=$),
  \item in addition boolean expressions can contain boolean operations: AND ($ \&\& $)  and OR ($||$). 
\end{itemize}

\subsubsection{Default output of relations}
For simple expressions which contain a relation in it, we generate output code adequate for the AND operator:
The relation $(a<b)$ is leading to following output:
\begin{quote}
\begin{tabbing}
erste \= Spaltes\=  Jump\= Spaltes \kill
\>CMP\>a,b\\
\>BGE\>\>F(alse Jump)\\
\end{tabbing}
\end{quote}

\subsubsection{Nested boolean expression}
We implemented different levels for nested boolean expressions.
Consider following expression: 
\begin{quote}
$\underbrace{(\underbrace{\underbrace{(a<b)}_{3}\&\&\underbrace{(b<c)}_{3}}_{2})||(\ldots)}_1 $
\end{quote}
The operator $ \&\& $ has only permission to edit already generated output code of the same level or of deeper levels. When a
boolean operation has edit a certain output code $x$ the level of $x$ will be decreased. When $ \&\& $ is finished (in the example above)
 all output code statements will be of the same level as $ \&\& $. This ensures that operations following the OR $||$ operator (where the
$\ldots$ are) can not edit previous output code.
That way we always need brackets for a valid sub-expression and $||$ does not bind stronger than $ \&\& $. We did not improve this since
we have limited time developing our compiler.

\subsubsection{Editing rules for boolean operators}
\begin{enumerate}
  \item (default)
  \begin{quote}
	\begin{tabbing}
	xxxxxxxxxSpalteserste \= Spaltes\=  Jump\= Spaltes \kill
	$(a<b)$\>CMP\>a,b\\
	\>BGE\>\>F(alse Jump)\\
	\end{tabbing}
	\end{quote}
OR changes the preceding control instruction when it has a false jump. 
It inverts the control instruction and changes the false into a true
  jump.
    \begin{quote}
	\begin{tabbing}
	xxxxxxxxxSpalteserste \= Spaltes\=  Jump\= Spaltes \kill
	$(a<b) || $\>CMP\>a,b\\
	\>BLT\>\>T(rue Jump)\\
	\end{tabbing}
	\end{quote}
\begin{verbatim}

\end{verbatim}
    \item
    
  \begin{quote}
	\begin{tabbing}
	xxxxxxxxxSpalteserste \= Spaltes\=  Jump\= Spaltes \kill
	$((a<b) \&\&  (b<d))$\>CMP\>a,b\\
	\>BGE\>\>F(alse Jump)\\
	\>CMP\>b,d\\
	\>BGE\>\>F(alse Jump)\\
	\end{tabbing}
	\end{quote}
      The operator OR changes the preceding control instruction at position x, as described above. In addition it changes all False
      instructions at position < x to the program counter next to $||$.  
  \begin{quote}
	\begin{tabbing}
	xxxxxxxxxSpalteserste \= Spaltes\=  Jump\= Spaltes \kill
	$((a<b) \&\&  (b<d))||$\>CMP\>a,b\\
	\>BGE\>\>next instruction after $||$\\
	\>CMP\>b,d\\
	\>BLT\>\>T(rue Jump)\\
	\end{tabbing}
	\end{quote}
\begin{verbatim}

\end{verbatim}

  \item   
  \begin{quote}
	\begin{tabbing}
	xxxxxxxxxSpalteserste \= Spaltes\=  Jump\= Spaltes \kill
	$((a<b) ||  (b<d))||$\>CMP\>a,b\\
	\>BLT\>\>T(rue Jump)\\
	\>CMP\>b,d\\
	\>BGE\>\>F(alse Jump)\\
	\end{tabbing}
	\end{quote}
The boolean operator OR does only change the precending control instruction as described in the first rule.
  \begin{quote}
	\begin{tabbing}
	xxxxxxxxxSpalteserste \= Spaltes\=  Jump\= Spaltes \kill
	$((a<b) ||  (b<d))||$\>CMP\>a,b\\
	\>BLT\>\>T(rue Jump)\\
	\>CMP\>b,d\\
	\>BLT\>\>T(rue) Jump)\\
	\end{tabbing}
	\end{quote}
\end{enumerate}
The AND operator works analogous, only true and false are inverted. The left false and true jumps which are not fixed at the end of the
expression will be fixed in the condition or repeated statements. Our rules works
really good, we can parse every expression independent of the complexity. We tested expressions with a level of 10 and more. \begin{small}
                                                          Idea copyright by team ComPiler :)
                                                          \end{small}
Worth notifying is that $\&\&$ works same as in Java when the first
expression is false the second one will not be evaluated. Analogous works $||$.
\subsection{Modules}
We support separate compilation and we can call functions of a foreign class when the class is being imported. After compiling each module
into a object file the linker will link them into a executable for the our virtual machine. Modules can contain global variables and methods.
 The order of the declarations is important, when the first method is being declared no global variable declaration can
 follow.


\subsection{Methods}
Methods are one of the most important tool for structuring programs. The methods we are able to compile can have parameters and a return
value. Our methods are able to hold local variables according to local hiding principle, see \ref{label_local_hiding}. The methods we
support must are public and static. The steps a method declaration goes through in our compiler:
\begin{itemize}
  \item When the parser recognize a method declaration the method gets an entry (cell $x$) in the global symbol table (\ref{labelSymboltable}).
  \item The return type will be recognized and stored as the method's type in the cell $x$. When a module serves as return
  type, then the module must have been imported. If no return type is declared a syntax error will be printed.
  \item Next, the parameters are parsed and for each parameter an entry in the methods separate symbol table is created. The offset of the
  parameters is a positive value. 
  \item The methods prologue is executed and the program counter of the prologue start is stored in cell $x$ of the global
  symboltable. The programcounter is needed for further method calls. 
  \item Next, method's body will being parsed. All new symbols, which are declared in the body are being stored in the sub- symbol list of
  the method. These local variables have a negative offset and have relative offsets to cell $x$ of the method.
  \item At the end of the method's body we calculate the size of the method. We sum the size of each element of the local symbol table and
  store the value into cell $x$ which resides in the global symbol table.
  \item The methods epilogue is executed.
  \item At this moment the method is ready to be called, recursive calls are possible too.
\end{itemize}
\begin{verbatim}

\end{verbatim}
\begin{quote}
Method's prologue: \\
\begin{itemize}
  \item push return address from the link register (R31) onto the stack
  \item push frame pointer (R29) onto the stack
  \item move frame pointer onto the current stack pointer
  \item decrease stack pointer with the size of the method (sum of local variables) 
\end{itemize}
\end{quote}
\begin{verbatim}

\end{verbatim}

\begin{quote}
Method's Epilogue: \\
\begin{itemize}
  \item move the stack pointer (R30) to the current frame pointer. Thats the advantage of the stack, we just move the stack pointer and the
  memory is free.
  \item pop, write back the old frame pointer
  \item pop, write back the old return address
  \item jump to the return address and continue
\end{itemize}
\end{quote}
\begin{quote}
\begin{verbatim}

\end{verbatim}
Method Call: \\
\begin{itemize}
  \item put parameters onto the stack
  \item branch to method
\end{itemize}

\end{quote}

\subsubsection{Main method}
The main method needs to be treated special because the program execution starts there. When we link modules we must have at least one
module which contains a main method. We can also deal with more modules containing a main method, because the linker needs the ``main''
module as argument. 

\subsubsection{Local hiding}
\label{label_local_hiding}
ComPiler offers local hiding of variables. That means, that procedures can
contain variables that cannot be seen outside the procedure. Even if there
exists a variable with the same name outside the procedure, these two don't
interfere.

\subsubsection{Memory organization of methods}
In the memory a method and its content is hold in activation frames. An activation frames is a special memory context used for a
method and its local variables.
The memory for the activation frames is organized as a stack. The SP indicates the next free memory and the FP the base address of the
current frame. When a method call happens, a new frame will be created and the frame pointer will be reassigned to the new frame. 


\subsection{Arrays}
\label{labelArrays}
In our compiler an array is represented as several variables of the specified type, as many as the size of the array is. We implemented arrays
in an early stage of our compiler, at this stage we did not have modules. The advantage of this implementation is that it is simple and
sufficient, the disadvantage is that we have no operations on arrays. \\
An array declaration has to be one this:
\begin{lstlisting}[caption={array declaration}]
dataType[] name = new dataType[expression];
dataType[] name = dataType[expression];
\end{lstlisting}
\paragraph{}When parsing an array declaration the type and size of the array will be stored in the symbol table. Arrays can be
local and global. To store new elements in the array they have to be, of course, be of the same type. In addition we check the
index bounds when writing or accessing an element of an array in compile time.  



