\section{The symbolfile}
The symbolfile (file-extension is .sym)is a sequential representation of the
symboltable. All entries
are representations of the symbols found in the sourcecode. \\
Because the symbolfile is just read one time during compilation, it cannot be
seen as a performance factor. So we decided to write the symbolfile in
xml-format as it is a good representation of objects. 
\subsection{Structure}
The symbolfile starts with a list of the so-called \emph{module anchors}. Every
module anchor represents a module that's imported. In xml-syntax this lookes
like that: \\
After that the actual symboltable representation starts. There are 3 different
elementtypes that can be described here:
\begin{itemize}
  \item variables
  \item arrays
  \item methods 
\end{itemize}
\subsubsection{variables}
A variable has 2 properties: 
\begin{itemize}
  \item one of the primitive datatypes like \emph{boolean}, \emph{int} or \emph{char}
  \item the name of the variable in the sourcecode
\end{itemize}
\subsubsection{arrays}
An array has 3 properties
\begin{itemize}
  \item one of the primitive datatypes like \emph{boolean}, \emph{int} or \emph{char}
  \item the name of the variable in the sourcecode
  \item the number of elements in the array
\end{itemize}
\subsubsection{methods}
An array has 3 properties and contains another symboltable with the parameters
of the method. 
\begin{itemize}
  \item the return value of the method as one of the primitive datatypes
  like \emph{boolean}, \emph{int} or \emph{char}
  \item the name of the method in the sourcecode
  \item the number of elements in the array
\end{itemize}
So the description of a method is a recursive search through the symboltable.