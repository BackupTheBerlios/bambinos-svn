\part{Linker}
The linker is a intermediate step between processing the sourcecode and executing a
binary. It combines objectfiles from the compiler to one binary for the virtual
machine. These objectfiles contain information for the linker how to combine
them to a binary. 

\section{The objectfile}
\label{objectfile}
The objectfile is a intermediate file created by the compiler, that already
contains generated code and a lot of meta-information. The linker reads this
objectfile and creates a binaryfile. \\
The reason for that intermediate step is, that the compiler cannot read sourcecode
from imported modules at compiletime. So it just writes a reference of that modulecode
in the objectfile. The linker however can read modules compiled separately
(modules are in objectfile format) and link that reference together with the
sourcecode and export it to a binaryfile.
\subsection{Structure}
\label{objectfile_structure}

\begin{tabular}{|c|}
\hline
magic word \\
\hline 
branch instruction to main method \\
\hline
lenght of offset table \\
\hline
offset table \\ \\
\hline
lenght of fixup table \\
\hline
fixup table \\ \\
\hline
opCode \\ \\
\hline 
\end{tabular}

\subsubsection{magic word} 
\label{magic_word}
The magic word identifies the objectfile. Its value has to be $0$ represented
by a 32 bit integer value. 
\subsubsection{branch instruction to main method}
\label{branch_to_main}
This is an integer-representation of the branch-instruction to the address of
the main-method in this objectfile. 
\subsubsection{length of the offset table}
\label{length_offset_table}
The length of the offset table as a 32 bit integer value. The unit of this value
is one 32 bit word. 
\subsubsection{offset table}
\label{offset_table}
In the offset table variable- or methodnames and their offsets in the current
module are saved. That means that every exported module element stands in the
offset table with its name and opCode offset in the modulefile.  \\
As an example we take the method \emph{print}:

\begin{center}
	\begin{tabular}{|c|c|c|c|}
		\hline
		P &  R & I & N \\
		\hline
		T &  = &  0 & 0 \\
		\hline 
		\multicolumn{4}{|c|}{value} \\
		\hline
		\multicolumn{4}{|c|}{\ldots} \\
		\hline
	\end{tabular}
\end{center}

The bytes until the equal-sign form the name of the symbol. The next bytes
are skipped so that only complete 32 bit words are read (in our example 2 bytes). 
The next 32 bit word is the offset of the instruction (opCode) for access to that 
element in the module. The length of the offset table ~\ref{length_offset_table}  
defines how often this operations are performed. 

\subsubsection{length of the fixup table}
\label{length_fixup_table}
The length of the fixup table as a 32 bit integer value. The unit of this value
is one 32 bit word. 
\subsubsection{fixup table}
\label{fixup_table}
In the fixup table the imported symbols are listed. That means their modulename,
their name and their offset in the module are stored. \\
As an example we take the \emph{print}-method from the module \emph{Util}.

\begin{center}
	\begin{tabular}{|c|c|c|c|}
		\hline
		U & T & I & L \\
		\hline
		. & P & R & I \\
		\hline
		N & T & = & 0 \\
		\hline 
		\multicolumn{4}{|c|}{value} \\
		\hline
		\multicolumn{4}{|c|}{\ldots} \\
		\hline
	\end{tabular}
\end{center}
The bytes from the beginning to the first dot form the modulename. The following
bytes to the equals-sign form the symbolname. The next bytes (in our example one
byte) are skipped so that only 32 bit words are read. \\
This reading-operation is performed until the number of words in the length of
the fixup table ~\ref{length_fixup_table} is read. 
