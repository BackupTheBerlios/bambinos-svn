\section{Memorymanagement in comPiler}
ComPiler compiles code for RISC-based architectures. It has 32 registers each
with size 32 Bit (4 Byte). Register 0 (R0) has always the value 0. Register 31
(R31) is reserved for return addresses. In a branch instruction the PC is stored
in R31. 

The instruction register (IR) holds the current instruction being executed. \newline 
The program counter (PC) contains the address of the instruction to be
fetched next. \newline
The stack pointer (SP) indicates the top element of a register-based stack. That
means that on some occasions registers are accessed like a stack and the SP
points out the next free register.  
\newline
The memory for the activation frames is organized like a stack. Each frame is an
entry. The SP indicates the next free memory and the FP the base address of the
current frame. 
\subsubsection{Organization of an activation frame}
An activation frame is a special memory context used for a procedure and its
local variables. It's created by a branch instruction (e.g. a procedure call). The base address of the activation 
frame is also the base address for all local variables declared here. Because this base address is highly 
important and one needs to access it efficently it's saved in the frame pointer (FP).
As the former PC was saved at the branch instruction, we use it as our return
address and save it in R31.



\subsection{local variables}
ComPiler offers local hiding of variables. That means, that procedures can
contain variables that cannot be seen outside the procedure. Even if there
exists a variable with the same name outside the procedure, these two don't
interfere. 
\subsubsection*{So what happens when a branch instruction occurs?}
When the instruction is executed, the PC is stored in R31. Then the code
generator jumps to the next instruction-address indicated by the branch instruction.
Now we have entered a special memory context for procedures (an activation frame). In this frame all memory is 
managed that is needed in the procedure
\newline
  
 


Local variables have the following properties: 
\begin{itemize}
  \item They have negative offsets to their baseaddresses
\end{itemize}


